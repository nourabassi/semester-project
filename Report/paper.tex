
% Default to the notebook output style

    


% Inherit from the specified cell style.




    \documentclass[journal,twocolumn]{IEEEtran}
\usepackage[noadjust]{cite}

    
    

    \usepackage[T1]{fontenc}
    % Nicer default font (+ math font) than Computer Modern for most use cases
    \usepackage{mathpazo}

    % Basic figure setup, for now with no caption control since it's done
    % automatically by Pandoc (which extracts ![](path) syntax from Markdown).
    \usepackage{graphicx}
    % We will generate all images so they have a width \maxwidth. This means
    % that they will get their normal width if they fit onto the page, but
    % are scaled down if they would overflow the margins.
    \makeatletter
    \def\maxwidth{\ifdim\Gin@nat@width>\linewidth\linewidth
    \else\Gin@nat@width\fi}
    \makeatother
    \let\Oldincludegraphics\includegraphics
    % Set max figure width to be 80% of text width, for now hardcoded.
    \renewcommand{\includegraphics}[1]{\Oldincludegraphics[width=.8\maxwidth]{#1}}
    % Ensure that by default, figures have no caption (until we provide a
    % proper Figure object with a Caption API and a way to capture that
    % in the conversion process - todo).
    \usepackage{caption}
    \DeclareCaptionLabelFormat{nolabel}{}
    \captionsetup{labelformat=nolabel}

    \usepackage{adjustbox} % Used to constrain images to a maximum size 
    \usepackage{xcolor} % Allow colors to be defined
    \usepackage{enumerate} % Needed for markdown enumerations to work
    \usepackage{geometry} % Used to adjust the document margins
    \usepackage{amsmath} % Equations
    \usepackage{amssymb} % Equations
    \usepackage{textcomp} % defines textquotesingle
    % Hack from http://tex.stackexchange.com/a/47451/13684:
    \AtBeginDocument{%
        \def\PYZsq{\textquotesingle}% Upright quotes in Pygmentized code
    }
    \usepackage{upquote} % Upright quotes for verbatim code
    \usepackage{eurosym} % defines \euro
    \usepackage[mathletters]{ucs} % Extended unicode (utf-8) support
    \usepackage[utf8x]{inputenc} % Allow utf-8 characters in the tex document
    \usepackage{fancyvrb} % verbatim replacement that allows latex
    \usepackage{grffile} % extends the file name processing of package graphics 
                         % to support a larger range 
    % The hyperref package gives us a pdf with properly built
    % internal navigation ('pdf bookmarks' for the table of contents,
    % internal cross-reference links, web links for URLs, etc.)
    \usepackage{hyperref}
    \usepackage{longtable} % longtable support required by pandoc >1.10
    \usepackage{booktabs}  % table support for pandoc > 1.12.2
    \usepackage[inline]{enumitem} % IRkernel/repr support (it uses the enumerate* environment)
    \usepackage[normalem]{ulem} % ulem is needed to support strikethroughs (\sout)
                                % normalem makes italics be italics, not underlines
    
\usepackage{mathptmx}
\usepackage{algorithm}\usepackage{algorithmic}


    
    
    % Colors for the hyperref package
    \definecolor{urlcolor}{rgb}{0,.145,.698}
    \definecolor{linkcolor}{rgb}{.71,0.21,0.01}
    \definecolor{citecolor}{rgb}{.12,.54,.11}

    % ANSI colors
    \definecolor{ansi-black}{HTML}{3E424D}
    \definecolor{ansi-black-intense}{HTML}{282C36}
    \definecolor{ansi-red}{HTML}{E75C58}
    \definecolor{ansi-red-intense}{HTML}{B22B31}
    \definecolor{ansi-green}{HTML}{00A250}
    \definecolor{ansi-green-intense}{HTML}{007427}
    \definecolor{ansi-yellow}{HTML}{DDB62B}
    \definecolor{ansi-yellow-intense}{HTML}{B27D12}
    \definecolor{ansi-blue}{HTML}{208FFB}
    \definecolor{ansi-blue-intense}{HTML}{0065CA}
    \definecolor{ansi-magenta}{HTML}{D160C4}
    \definecolor{ansi-magenta-intense}{HTML}{A03196}
    \definecolor{ansi-cyan}{HTML}{60C6C8}
    \definecolor{ansi-cyan-intense}{HTML}{258F8F}
    \definecolor{ansi-white}{HTML}{C5C1B4}
    \definecolor{ansi-white-intense}{HTML}{A1A6B2}

    % commands and environments needed by pandoc snippets
    % extracted from the output of `pandoc -s`
    \providecommand{\tightlist}{%
      \setlength{\itemsep}{0pt}\setlength{\parskip}{0pt}}
    \DefineVerbatimEnvironment{Highlighting}{Verbatim}{commandchars=\\\{\}}
    % Add ',fontsize=\small' for more characters per line
    \newenvironment{Shaded}{}{}
    \newcommand{\KeywordTok}[1]{\textcolor[rgb]{0.00,0.44,0.13}{\textbf{{#1}}}}
    \newcommand{\DataTypeTok}[1]{\textcolor[rgb]{0.56,0.13,0.00}{{#1}}}
    \newcommand{\DecValTok}[1]{\textcolor[rgb]{0.25,0.63,0.44}{{#1}}}
    \newcommand{\BaseNTok}[1]{\textcolor[rgb]{0.25,0.63,0.44}{{#1}}}
    \newcommand{\FloatTok}[1]{\textcolor[rgb]{0.25,0.63,0.44}{{#1}}}
    \newcommand{\CharTok}[1]{\textcolor[rgb]{0.25,0.44,0.63}{{#1}}}
    \newcommand{\StringTok}[1]{\textcolor[rgb]{0.25,0.44,0.63}{{#1}}}
    \newcommand{\CommentTok}[1]{\textcolor[rgb]{0.38,0.63,0.69}{\textit{{#1}}}}
    \newcommand{\OtherTok}[1]{\textcolor[rgb]{0.00,0.44,0.13}{{#1}}}
    \newcommand{\AlertTok}[1]{\textcolor[rgb]{1.00,0.00,0.00}{\textbf{{#1}}}}
    \newcommand{\FunctionTok}[1]{\textcolor[rgb]{0.02,0.16,0.49}{{#1}}}
    \newcommand{\RegionMarkerTok}[1]{{#1}}
    \newcommand{\ErrorTok}[1]{\textcolor[rgb]{1.00,0.00,0.00}{\textbf{{#1}}}}
    \newcommand{\NormalTok}[1]{{#1}}
    
    % Additional commands for more recent versions of Pandoc
    \newcommand{\ConstantTok}[1]{\textcolor[rgb]{0.53,0.00,0.00}{{#1}}}
    \newcommand{\SpecialCharTok}[1]{\textcolor[rgb]{0.25,0.44,0.63}{{#1}}}
    \newcommand{\VerbatimStringTok}[1]{\textcolor[rgb]{0.25,0.44,0.63}{{#1}}}
    \newcommand{\SpecialStringTok}[1]{\textcolor[rgb]{0.73,0.40,0.53}{{#1}}}
    \newcommand{\ImportTok}[1]{{#1}}
    \newcommand{\DocumentationTok}[1]{\textcolor[rgb]{0.73,0.13,0.13}{\textit{{#1}}}}
    \newcommand{\AnnotationTok}[1]{\textcolor[rgb]{0.38,0.63,0.69}{\textbf{\textit{{#1}}}}}
    \newcommand{\CommentVarTok}[1]{\textcolor[rgb]{0.38,0.63,0.69}{\textbf{\textit{{#1}}}}}
    \newcommand{\VariableTok}[1]{\textcolor[rgb]{0.10,0.09,0.49}{{#1}}}
    \newcommand{\ControlFlowTok}[1]{\textcolor[rgb]{0.00,0.44,0.13}{\textbf{{#1}}}}
    \newcommand{\OperatorTok}[1]{\textcolor[rgb]{0.40,0.40,0.40}{{#1}}}
    \newcommand{\BuiltInTok}[1]{{#1}}
    \newcommand{\ExtensionTok}[1]{{#1}}
    \newcommand{\PreprocessorTok}[1]{\textcolor[rgb]{0.74,0.48,0.00}{{#1}}}
    \newcommand{\AttributeTok}[1]{\textcolor[rgb]{0.49,0.56,0.16}{{#1}}}
    \newcommand{\InformationTok}[1]{\textcolor[rgb]{0.38,0.63,0.69}{\textbf{\textit{{#1}}}}}
    \newcommand{\WarningTok}[1]{\textcolor[rgb]{0.38,0.63,0.69}{\textbf{\textit{{#1}}}}}
    
    
    % Define a nice break command that doesn't care if a line doesn't already
    % exist.
    \def\br{\hspace*{\fill} \\* }
    % Math Jax compatability definitions
    \def\gt{>}
    \def\lt{<}
    % Document parameters
    \title{paper}
    
    
    

    % Pygments definitions
    
\makeatletter
\def\PY@reset{\let\PY@it=\relax \let\PY@bf=\relax%
    \let\PY@ul=\relax \let\PY@tc=\relax%
    \let\PY@bc=\relax \let\PY@ff=\relax}
\def\PY@tok#1{\csname PY@tok@#1\endcsname}
\def\PY@toks#1+{\ifx\relax#1\empty\else%
    \PY@tok{#1}\expandafter\PY@toks\fi}
\def\PY@do#1{\PY@bc{\PY@tc{\PY@ul{%
    \PY@it{\PY@bf{\PY@ff{#1}}}}}}}
\def\PY#1#2{\PY@reset\PY@toks#1+\relax+\PY@do{#2}}

\expandafter\def\csname PY@tok@w\endcsname{\def\PY@tc##1{\textcolor[rgb]{0.73,0.73,0.73}{##1}}}
\expandafter\def\csname PY@tok@c\endcsname{\let\PY@it=\textit\def\PY@tc##1{\textcolor[rgb]{0.25,0.50,0.50}{##1}}}
\expandafter\def\csname PY@tok@cp\endcsname{\def\PY@tc##1{\textcolor[rgb]{0.74,0.48,0.00}{##1}}}
\expandafter\def\csname PY@tok@k\endcsname{\let\PY@bf=\textbf\def\PY@tc##1{\textcolor[rgb]{0.00,0.50,0.00}{##1}}}
\expandafter\def\csname PY@tok@kp\endcsname{\def\PY@tc##1{\textcolor[rgb]{0.00,0.50,0.00}{##1}}}
\expandafter\def\csname PY@tok@kt\endcsname{\def\PY@tc##1{\textcolor[rgb]{0.69,0.00,0.25}{##1}}}
\expandafter\def\csname PY@tok@o\endcsname{\def\PY@tc##1{\textcolor[rgb]{0.40,0.40,0.40}{##1}}}
\expandafter\def\csname PY@tok@ow\endcsname{\let\PY@bf=\textbf\def\PY@tc##1{\textcolor[rgb]{0.67,0.13,1.00}{##1}}}
\expandafter\def\csname PY@tok@nb\endcsname{\def\PY@tc##1{\textcolor[rgb]{0.00,0.50,0.00}{##1}}}
\expandafter\def\csname PY@tok@nf\endcsname{\def\PY@tc##1{\textcolor[rgb]{0.00,0.00,1.00}{##1}}}
\expandafter\def\csname PY@tok@nc\endcsname{\let\PY@bf=\textbf\def\PY@tc##1{\textcolor[rgb]{0.00,0.00,1.00}{##1}}}
\expandafter\def\csname PY@tok@nn\endcsname{\let\PY@bf=\textbf\def\PY@tc##1{\textcolor[rgb]{0.00,0.00,1.00}{##1}}}
\expandafter\def\csname PY@tok@ne\endcsname{\let\PY@bf=\textbf\def\PY@tc##1{\textcolor[rgb]{0.82,0.25,0.23}{##1}}}
\expandafter\def\csname PY@tok@nv\endcsname{\def\PY@tc##1{\textcolor[rgb]{0.10,0.09,0.49}{##1}}}
\expandafter\def\csname PY@tok@no\endcsname{\def\PY@tc##1{\textcolor[rgb]{0.53,0.00,0.00}{##1}}}
\expandafter\def\csname PY@tok@nl\endcsname{\def\PY@tc##1{\textcolor[rgb]{0.63,0.63,0.00}{##1}}}
\expandafter\def\csname PY@tok@ni\endcsname{\let\PY@bf=\textbf\def\PY@tc##1{\textcolor[rgb]{0.60,0.60,0.60}{##1}}}
\expandafter\def\csname PY@tok@na\endcsname{\def\PY@tc##1{\textcolor[rgb]{0.49,0.56,0.16}{##1}}}
\expandafter\def\csname PY@tok@nt\endcsname{\let\PY@bf=\textbf\def\PY@tc##1{\textcolor[rgb]{0.00,0.50,0.00}{##1}}}
\expandafter\def\csname PY@tok@nd\endcsname{\def\PY@tc##1{\textcolor[rgb]{0.67,0.13,1.00}{##1}}}
\expandafter\def\csname PY@tok@s\endcsname{\def\PY@tc##1{\textcolor[rgb]{0.73,0.13,0.13}{##1}}}
\expandafter\def\csname PY@tok@sd\endcsname{\let\PY@it=\textit\def\PY@tc##1{\textcolor[rgb]{0.73,0.13,0.13}{##1}}}
\expandafter\def\csname PY@tok@si\endcsname{\let\PY@bf=\textbf\def\PY@tc##1{\textcolor[rgb]{0.73,0.40,0.53}{##1}}}
\expandafter\def\csname PY@tok@se\endcsname{\let\PY@bf=\textbf\def\PY@tc##1{\textcolor[rgb]{0.73,0.40,0.13}{##1}}}
\expandafter\def\csname PY@tok@sr\endcsname{\def\PY@tc##1{\textcolor[rgb]{0.73,0.40,0.53}{##1}}}
\expandafter\def\csname PY@tok@ss\endcsname{\def\PY@tc##1{\textcolor[rgb]{0.10,0.09,0.49}{##1}}}
\expandafter\def\csname PY@tok@sx\endcsname{\def\PY@tc##1{\textcolor[rgb]{0.00,0.50,0.00}{##1}}}
\expandafter\def\csname PY@tok@m\endcsname{\def\PY@tc##1{\textcolor[rgb]{0.40,0.40,0.40}{##1}}}
\expandafter\def\csname PY@tok@gh\endcsname{\let\PY@bf=\textbf\def\PY@tc##1{\textcolor[rgb]{0.00,0.00,0.50}{##1}}}
\expandafter\def\csname PY@tok@gu\endcsname{\let\PY@bf=\textbf\def\PY@tc##1{\textcolor[rgb]{0.50,0.00,0.50}{##1}}}
\expandafter\def\csname PY@tok@gd\endcsname{\def\PY@tc##1{\textcolor[rgb]{0.63,0.00,0.00}{##1}}}
\expandafter\def\csname PY@tok@gi\endcsname{\def\PY@tc##1{\textcolor[rgb]{0.00,0.63,0.00}{##1}}}
\expandafter\def\csname PY@tok@gr\endcsname{\def\PY@tc##1{\textcolor[rgb]{1.00,0.00,0.00}{##1}}}
\expandafter\def\csname PY@tok@ge\endcsname{\let\PY@it=\textit}
\expandafter\def\csname PY@tok@gs\endcsname{\let\PY@bf=\textbf}
\expandafter\def\csname PY@tok@gp\endcsname{\let\PY@bf=\textbf\def\PY@tc##1{\textcolor[rgb]{0.00,0.00,0.50}{##1}}}
\expandafter\def\csname PY@tok@go\endcsname{\def\PY@tc##1{\textcolor[rgb]{0.53,0.53,0.53}{##1}}}
\expandafter\def\csname PY@tok@gt\endcsname{\def\PY@tc##1{\textcolor[rgb]{0.00,0.27,0.87}{##1}}}
\expandafter\def\csname PY@tok@err\endcsname{\def\PY@bc##1{\setlength{\fboxsep}{0pt}\fcolorbox[rgb]{1.00,0.00,0.00}{1,1,1}{\strut ##1}}}
\expandafter\def\csname PY@tok@kc\endcsname{\let\PY@bf=\textbf\def\PY@tc##1{\textcolor[rgb]{0.00,0.50,0.00}{##1}}}
\expandafter\def\csname PY@tok@kd\endcsname{\let\PY@bf=\textbf\def\PY@tc##1{\textcolor[rgb]{0.00,0.50,0.00}{##1}}}
\expandafter\def\csname PY@tok@kn\endcsname{\let\PY@bf=\textbf\def\PY@tc##1{\textcolor[rgb]{0.00,0.50,0.00}{##1}}}
\expandafter\def\csname PY@tok@kr\endcsname{\let\PY@bf=\textbf\def\PY@tc##1{\textcolor[rgb]{0.00,0.50,0.00}{##1}}}
\expandafter\def\csname PY@tok@bp\endcsname{\def\PY@tc##1{\textcolor[rgb]{0.00,0.50,0.00}{##1}}}
\expandafter\def\csname PY@tok@fm\endcsname{\def\PY@tc##1{\textcolor[rgb]{0.00,0.00,1.00}{##1}}}
\expandafter\def\csname PY@tok@vc\endcsname{\def\PY@tc##1{\textcolor[rgb]{0.10,0.09,0.49}{##1}}}
\expandafter\def\csname PY@tok@vg\endcsname{\def\PY@tc##1{\textcolor[rgb]{0.10,0.09,0.49}{##1}}}
\expandafter\def\csname PY@tok@vi\endcsname{\def\PY@tc##1{\textcolor[rgb]{0.10,0.09,0.49}{##1}}}
\expandafter\def\csname PY@tok@vm\endcsname{\def\PY@tc##1{\textcolor[rgb]{0.10,0.09,0.49}{##1}}}
\expandafter\def\csname PY@tok@sa\endcsname{\def\PY@tc##1{\textcolor[rgb]{0.73,0.13,0.13}{##1}}}
\expandafter\def\csname PY@tok@sb\endcsname{\def\PY@tc##1{\textcolor[rgb]{0.73,0.13,0.13}{##1}}}
\expandafter\def\csname PY@tok@sc\endcsname{\def\PY@tc##1{\textcolor[rgb]{0.73,0.13,0.13}{##1}}}
\expandafter\def\csname PY@tok@dl\endcsname{\def\PY@tc##1{\textcolor[rgb]{0.73,0.13,0.13}{##1}}}
\expandafter\def\csname PY@tok@s2\endcsname{\def\PY@tc##1{\textcolor[rgb]{0.73,0.13,0.13}{##1}}}
\expandafter\def\csname PY@tok@sh\endcsname{\def\PY@tc##1{\textcolor[rgb]{0.73,0.13,0.13}{##1}}}
\expandafter\def\csname PY@tok@s1\endcsname{\def\PY@tc##1{\textcolor[rgb]{0.73,0.13,0.13}{##1}}}
\expandafter\def\csname PY@tok@mb\endcsname{\def\PY@tc##1{\textcolor[rgb]{0.40,0.40,0.40}{##1}}}
\expandafter\def\csname PY@tok@mf\endcsname{\def\PY@tc##1{\textcolor[rgb]{0.40,0.40,0.40}{##1}}}
\expandafter\def\csname PY@tok@mh\endcsname{\def\PY@tc##1{\textcolor[rgb]{0.40,0.40,0.40}{##1}}}
\expandafter\def\csname PY@tok@mi\endcsname{\def\PY@tc##1{\textcolor[rgb]{0.40,0.40,0.40}{##1}}}
\expandafter\def\csname PY@tok@il\endcsname{\def\PY@tc##1{\textcolor[rgb]{0.40,0.40,0.40}{##1}}}
\expandafter\def\csname PY@tok@mo\endcsname{\def\PY@tc##1{\textcolor[rgb]{0.40,0.40,0.40}{##1}}}
\expandafter\def\csname PY@tok@ch\endcsname{\let\PY@it=\textit\def\PY@tc##1{\textcolor[rgb]{0.25,0.50,0.50}{##1}}}
\expandafter\def\csname PY@tok@cm\endcsname{\let\PY@it=\textit\def\PY@tc##1{\textcolor[rgb]{0.25,0.50,0.50}{##1}}}
\expandafter\def\csname PY@tok@cpf\endcsname{\let\PY@it=\textit\def\PY@tc##1{\textcolor[rgb]{0.25,0.50,0.50}{##1}}}
\expandafter\def\csname PY@tok@c1\endcsname{\let\PY@it=\textit\def\PY@tc##1{\textcolor[rgb]{0.25,0.50,0.50}{##1}}}
\expandafter\def\csname PY@tok@cs\endcsname{\let\PY@it=\textit\def\PY@tc##1{\textcolor[rgb]{0.25,0.50,0.50}{##1}}}

\def\PYZbs{\char`\\}
\def\PYZus{\char`\_}
\def\PYZob{\char`\{}
\def\PYZcb{\char`\}}
\def\PYZca{\char`\^}
\def\PYZam{\char`\&}
\def\PYZlt{\char`\<}
\def\PYZgt{\char`\>}
\def\PYZsh{\char`\#}
\def\PYZpc{\char`\%}
\def\PYZdl{\char`\$}
\def\PYZhy{\char`\-}
\def\PYZsq{\char`\'}
\def\PYZdq{\char`\"}
\def\PYZti{\char`\~}
% for compatibility with earlier versions
\def\PYZat{@}
\def\PYZlb{[}
\def\PYZrb{]}
\makeatother


    % Exact colors from NB
    \definecolor{incolor}{rgb}{0.0, 0.0, 0.5}
    \definecolor{outcolor}{rgb}{0.545, 0.0, 0.0}



    
    % Prevent overflowing lines due to hard-to-break entities
    \sloppy 
    % Setup hyperref package
    \hypersetup{
      breaklinks=true,  % so long urls are correctly broken across lines
      colorlinks=true,
      urlcolor=urlcolor,
      linkcolor=linkcolor,
      citecolor=citecolor,
      }
    % Slightly bigger margins than the latex defaults
    
    \geometry{verbose,tmargin=1in,bmargin=1in,lmargin=1in,rmargin=1in}
    
    

    \begin{document}
    
    
    
\renewcommand{\figurename}{Figure}
\captionsetup{labelformat=simple}\title{Bibliometric Analysis of ICLS and CSCL research}
\author{\IEEEauthorblockN{Léonore Valentine Guillain, Nour Ghalia Abassi}\par
    \IEEEauthorblockA{EPFL}
}
\maketitle

    
    

    

\begin{abstract}
    \#TODO write some sentence about general motivation for bibliometric
analysis We explore ways to extract information from ICLS and CSCL
papers to gain an understanding about the community formed by conference
participants and what research is being conducted within the community.
After describing how to extract data from the original dataset in
suitable format for analysis we perform an exploratory data analysis
after which we derive insights into the community using social network
analysis and {[}Nour: NLP etc{]}. \#TODO write a few centences about the
most central findings
\end{abstract}
    \hypertarget{introduction}{%
\section{Introduction}\label{introduction}}

Lorem ipsum dolor sit amet, consectetur adipiscing elit, sed do eiusmod
tempor incididunt ut labore et dolore magna aliqua. Ut enim ad minim
veniam, quis nostrud exercitation ullamco laboris nisi ut aliquip ex ea
commodo consequat. Duis aute irure dolor in reprehenderit in voluptate
velit esse cillum dolore eu fugiat nulla pariatur. Excepteur sint
occaecat cupidatat non proident, sunt in culpa qui officia deserunt
mollit anim id est laborum.

    \hypertarget{data-processing}{%
\section{Data Processing}\label{data-processing}}

    As a starting point we have papers of 4 years spanning 2015 to 2018 of
the proceedings of ICLS, which is held in even years, respectively CSCL,
held in odd years. The overall number of papers per year can be observed
in figure \(\ref{fig1}\). The provided raw dataset consists of papers in
pdf format as well as associated metadata in xml format. To extract
information from the dataset we first need to convert the data a format
that is more adapt to the task of data exploration and handling with a
variety of external libraries. We store the data we extract from the
provided dataset in csv format, as this makes it accessible to any
scientist and is unlikely to become an obsolete any time soon.


    \begin{figure}
        \begin{center}\adjustimage{max size={\linewidth}{0.4\paperheight}}{paper_files/paper_8_0.png}\end{center}
        \caption{Distribution of papers published over the years}
        \label{fig1}
    \end{figure}
    
    \hypertarget{metadata}{%
\subsection{Metadata}\label{metadata}}

We extract the xml metadata --using the lxmltree python library-- into a
csv file containing the information associated to different tags in its
columns. The provided xml does not exactly follow the xml standard,
hence we handle errors by iteratively parsing the xml tree. We then add
columns that reference the name of the source file and the order in
which an author name occurs in the xml file. From the associated
citation string we add in a shortened version of author names, and the
``author names (year) title'' string that we later use to identify a
paper. Having extracted the metadata we then clean it by unifying names
as slight spelling differences exist within the same author name. This
is done by splitting up names based on whitespace and commas. If the
intersection of the set of substrings of two distinct names contains at
least 2 names of length at least 3 we consider these two names to be
identical. Using this we reduce the number of authors in the dataset
from 1951 down to 1879. To unify the names further, we need to make sure
to not overmerge by labeling to different authors as having the same
name. We do this by constructing the graph of co-authorship and then
only considering names of authors that are in the same neighborhood of a
node on the graph. Then we use the difflib library to check for
similarity of two names and merge them if they reach a the threshold of
0.8. This method again reduces the number of authors in the dataset from
1879 to 1951.

    \hypertarget{handling-of-pdf-data}{%
\subsection{Handling of pdf data}\label{handling-of-pdf-data}}

To extract information from the papers themselves we need to first
extract the text from the pdfs. Various parsers performing this task
exist. However, many parsers {[}pdfminer, PyPDF2{]} that claim to be
able to extract the text as well as metadata from papers have not been
kept up to date, which leads to issues with dependencies. We found that
the most reliable parser is poppler https://poppler.freedesktop.org/.

We note that due to inconsistencies with numbering, some of the pdfs
were not extracted correctly from the proceedings. We make sure to
handle these cases.

The advantage of papers is that they are structured fairly regularly.
From ISLS\_Author\_Guidelines\_Oct2014.pdf we know papers the papers in
our dataset should be structured. We use this to our advantage when
extracting information from the paper text. As per
{[}ISLS\_Author\_Guidelines\_Oct2014{]} the document should be
structured in the following way: 1. Title 2. Abstract 3. Keywords 4.
Body of submission (using heading levels 1, 2 and 3 as necessary) 5.
Endnotes (if needed) 6. References 7. Acknowledgements (if needed)

\hypertarget{extracting-the-papers-cited-by-a-paper}{%
\subsection{Extracting the papers cited by a
paper}\label{extracting-the-papers-cited-by-a-paper}}

We know that each valid paper should contain a reference section at the
end listing all the references cited in the paper. While there are tools
being developed to extract the reference section directly from pdfs,
notably {[}scholarcy{]}, we found that none of them produced
satisfactory results as of the time writing this report. Hence we resort
the using regular expressions and our knowledge about how references
should be presented within the paper.

To get the reference section we use a regular expression to find the
`Reference' header and cut of at the end if an Acknowledgement or
Appendix section is present. Python has two modules that implement
regular expressions re and regex. We used regex as it allows to better
handle unicode strings by having unicode expressions, which frequently
occur.

To get the individual reference we tested many ways to split up the
lines, but ultimately, the most robust way was to split up the section
at each new line and the glue the references back together if the lines
satisfy a set of conditions.

The underlying idea is that APA references, the standard referencing
format, all have the same initial structure: Authors (year) title, other
identifying information. We thus check that the beginning of each
reference contains ``Authors (year)''. It is relatively easy to check
for this using regular expressions as Author names and year have a given
structure. If a string does not contain such a section, then it must be
part of an other reference, which is lies before in the list of
reference substrings.

Consider that the references strings are structured the following way
{[}\ldots{}., reference\_part\_i, reference\_part\_i+1, \ldots{}.{]}
Algorithm: 1. Merge names that were cut up. We detect them by checking
against the regular expression: \_\_ 2. Check ending for invalid
sentence endings: reference\_part\_i ends in comma, :, whitespace
uppercase followed by point. If condition holds reference\_part\_i+1 3.
Check beginning for invalid reference starts: not starting in a letter.
Merge with reference\_part\_i if condition holds. 4. Check if a invalid
reference\_part\_i+1 consists only of the authors names. If so Merge
with reference\_part\_i 5. Move up all sentences that do not contain
author names / start of a citation at the beginning 6. Move up all
sentences that do not have a valid citation start

After this we validate the returned split by looking at the overall
length and the initials, as well as checking that a random sample
returns satisfactory splitting accuracy. We find that the above method
of extracting the references produces good results for proper APA
citations and on APA citations outperforms Scolarcy.

    \hypertarget{extracting-institution-country}{%
\subsection{Extracting institution \&
country}\label{extracting-institution-country}}

Now we consider the information present in the section before the
abstract, which contains the authors names, their email and the
institution they are affiliated with. Affiliation with institution can
give us a lot of insight where people are working and gives use a way
estimate where they are from. To go from this section to an exact
affiliation with an institution a multifaceted approach was used.

\hypertarget{approach-1}{%
\subsubsection{Approach 1}\label{approach-1}}

The most robust way to get a persons institution affiliation is to use
their email. Emails can easily be extracted from the header section
using a regular expression and given a domain-university mapping {[}{]}
give us an onto mapping from domain to institution. The additional
advantage of this method is that using the additional metadata
associated to each university in the mapping, we also access information
about the country of the university.

\hypertarget{approach-2}{%
\subsubsection{Approach 2}\label{approach-2}}

The header section itself contains information on the institution, but
as different was to list authors are accepted, notably two ways {[}fig
\_ a and b{]}. Hence, we first detect the structure of the header
section before extracting the association. We tag the substring present
in this section using a combination of named entity detection, using the
implementation provided by spacy, simple matching of the names of
authors and institutions we extracted from the metadata respectively
from Approach 1 and tagging sections containing ``emails'' using regex ,
to detect where in the header the email is. Then, taking the tagged
section, we detect which subsections correspond to which
author-affiliation listing type, and extract the author email -
institution mapping accordingly.

Approach 2 allows to match many more institutions to authors than
Approach 1 as not every institution is present in the mapping and many
participants use gmail and similar providers as their official
correspondence email we are unable to acquire this information for many
participants. But, Approach 2 may contain differences in spelling in the
institution name, which need to be merged. Thus we combine both
approaches, and use the corresponding institutions in Approach 1 to
unify the institutions obtained in Approach 2.

As institution affiliation does not allow us to automatically associate
a country to the institution, we need to get this through additional
means. A first step is to use the top level domains present in the
emails to associate a country of origin to the participants \& their
associated institutions.

Alongside this, to plot the community we augment the dataset with
geolocation information, having this information we can then easily
associate a country to the institution.

As there are a variety of way in which the name of an institution could
be written --consider EPFL, EPF Lausanne-- it was found that crawling
the english, french and german version of Wikipedia (not using the API,
or the API wrapper, as this does not give access to the desired
information) and using beautiful soup the get the coordinates present in
the article was a quick and accurate way to get the location as the
website directs to the relevant article on the institution even when
misspellings are present. But not all institutions, even well known ones
such as Tokyo University, have their coordinates listed on their
Wikipedia. To get the remaining locations we use the geopy, which is a
client for geocoding web services. These results however are more error
prone and will always return a result. Then, to get the country we can
then again use geopy. The resulting data containing paper identifier,
the authors email, it's associated domain, order in which the author was
cited, the institution, country, Longitude and Latitude are saved to
csv. We use the paper identifier and author order to associate this data
with the metadata to link information about authors with the papers they
authored.

    \hypertarget{keywords-extraction}{%
\subsection{Keywords extraction}\label{keywords-extraction}}

    \hypertarget{text-preprocessing}{%
\subsubsection{Text preprocessing}\label{text-preprocessing}}

    From the pdf parser some of the documents were corrupted with some weird
caracters such as , white space strings

\begin{itemize}
\item
  We used \textbf{\emph{beautiful soup}} and \textbf{\emph{string}}
  libraries to only keep the text that we needed without any special
  character.
\item
  we applied \textbf{\emph{regular expression}} to remove numbers
  because we thought that numbers were not that important.
\item
  We put all the words to lower cases
\item
  We removed the words that were connecting parts of a sentence rather
  than showing subjects, objects or intent. For us these words were not
  that important and were only going to increase the bag of words length
\item
  In some methods we used \textbf{\emph{Lemmatization}} and
  \textbf{\emph{stemming}} and in other methods we didn't because we
  wanted to do a mapping with \textbf{\emph{GloVe}} vocabulary which we
  will explain later on.
\end{itemize}

\textbf{\emph{Lemmatization}} is a word variations to the root of the
word (e.g.~working, works, worked changed to work).
\textbf{\emph{Stemming}} is the process of reducing inflected and
derived words to their word stem, base or root form---generally a
written word form As the use of the stop words to reduce the BOW (bag of
words size we also used lemmatization and stemming to be able to reduce
the number of words without taking off any information)

    \textbf{\emph{Lemmatization}} and different form of
\textbf{\emph{Stemming}} were only used with \textbf{\emph{TF-IDF}}
while doing the Keywords extraction, otherwise we prefered keeping the
words as they are because we were doing a mapping with a pretrained
stanford word embedding and we did not wand to loose interpretability.

    For the Keywords extraction we used three methods. The first one whe is
a trivial one is to search in the pdf itself for keywords and extract
them, this method is the most accurate but we found out that for 492
papers we had no keywords included. So we deciced to use two methods for
keywords extraction. The first one is \textbf{\emph{TF-IDF}} and the
second one is \textbf{\emph{Rake}}.

    \begin{itemize}
\tightlist
\item
  \textbf{\emph{TF-IDF}}: Tf--idf is one of the most popular
  term-weighting schemes today, in information retrieval, tf--idf or
  TFIDF, short for term frequency--inverse document frequency, is a
  numerical statistic that is intended to reflect how important a word
  is to a document in a collection or corpus. It is often used as a
  weighting factor in searches of information retrieval, text mining,
  and user modeling. The tf--idf value increases proportionally to the
  number of times a word appears in the document and is offset by the
  number of documents in the corpus that contain the word, which helps
  to adjust for the fact that some words appear more frequently in
  general.
\end{itemize}

    With TF-IDF we sorted for each document the words with the highest score
and then selected the the 7 first words for each document.

    \begin{itemize}
\tightlist
\item
  \textbf{\emph{Rake}}: Rake stands for Rapid Automatic Keyword
  Extraction. It is an existing python implementation that uses the NLTK
  toolkit for the score calculations.
\end{itemize}

\textbf{\emph{How it works}} : * It splits the text into sentences and
generates some words as candidates.\\
* Various punctuation signs will be treated as sentence boundaries. *
All words listed in the stopwords file will be treated as phrase
boundaries. This helps generate candidates that consist of one or more
non-stopwords. However ,it won't work in cases where the stopword is
part of the phrase. For example, `new' is listed in RAKE's stopword
list. This means that neither `New York' nor `New Zealand' can be ever a
keyword. * For each keyword candidate generated, the algorithm computes
the score of the pretended keyword, which is the sum of the scores for
each of its words (if it's a composed word). The words are scored
according to their frequency and the typical length of a candidate
phrase in which they appear. * The last step is ranking the scores ansd
selecting the ones with the highest score

    \textbf{\emph{Keywords in text}}: Select from the text the keywords
already given by the authors of the paper.

    Once we have all the keywords from the three methods, we tried to select
the most accurate keywords. So we created a method that selects the
intersection of TF-IDF keywords and the Rake keywords with the keywords
of the authors (if there are any keywords in the text), if the
intersection is empty and we have the keywords in the text we only take
the keywords of the text. If we don't have any keywords we take the
intersection of the TF-IDF and the Rake. If again, the intersection is
empty then we only take the TF-IDF keywords because they were closer to
the text keywords than the Rake Keywords.

    Once we have our final keywords, we created a method that goes into the
metadata and put assign to every document a publication year.


    Since we don't have the same number of documents for each year we needed
to scale over the years and use the percentage of appearence of a word.
So after groupping the words by year and scaling them we found graph
below

\texttt{\color{outcolor}Out[{\color{outcolor}33}]:}
    
    \begin{figure}
        \begin{center}\adjustimage{max size={\linewidth}{0.4\paperheight}}{paper_files/paper_25_0.png}\end{center}
        \caption{}
        \label{}
    \end{figure}
    

    We wanted then to know what are the trends of words for each specific
year, so we selected distinct words for each year, it means that we
selected for each year the words that were present for that year and
never for the others. We get the graph below. As we can see from the
graph, the most used word in the papers, more than 10\% of 2015 papers
uses the word `peer', more than 6\% of the papers uses the words
`building', `embodied' in 2016. In 2017 the most frequent keyword on is
`reponsive' and in 2018 `tools' is present with more than 6\%.

    \hypertarget{methodology}{%
\section{Methodology}\label{methodology}}

    \hypertarget{document-clustering}{%
\subsection{Document Clustering}\label{document-clustering}}

    \hypertarget{illustrative-example}{%
\section{Illustrative Example}\label{illustrative-example}}

For the purposes of illustration, consider a damped harmonic oscillator
driven by a sinusoidal signal: \begin{equation}
\frac{\mathrm{d}^2x}{\mathrm{d}t^2} + 2\zeta\omega_0\frac{\mathrm{d}x}{\mathrm{d}t} + \omega_0^2x = A\sin(\omega t), \label{eq:problem}
\end{equation} \noindent where \(x(t)\) is the quantity of interest,
\(\zeta\) is the damping constant, \(\omega_0\) is the undamped resonant
frequency, and \(A\sin(\omega t)\) is the driving sinusoidal signal. The
steady state solution to \eqref{eq:problem} is given by:
\begin{eqnarray}
x(t) &=& \frac{A}{Z_m\omega}\sin(\omega t + \phi), \label{eq:soln}\\
Z_m &=& \sqrt{(2\omega_0\zeta)^2+\frac{(\omega^2-\omega_0^2)^2}{\omega^2}}, \nonumber\\
\phi &=& \arctan\left(\frac{2\omega\omega_0\zeta}{\omega^2-\omega_0^2}\right). \nonumber
\end{eqnarray}


    
    




    \begin{figure}
        \begin{center}\adjustimage{max size={\linewidth}{0.4\paperheight}}{paper_files/paper_34_0.png}\end{center}
        \caption{Variation of steady-state oscillation amplitude with driving frequency for an oscillator with undamped frequency of 100 Hz, a driving amplitude $A = 10^6$, and various damping constants.}
        \label{figtest}
    \end{figure}
    
    In Figure \ref{fig1} we plot the steady state amplitude of a 100 Hz
oscillator as a function of driving frequency. We see that the
oscillator exhibits a strong resonance for small values of \(\zeta\),
and that the resonance frequency is slightly lower than the driving
frequency of 100 Hz.

\begin{algorithm}
\caption{Calculate $y = x^n + z^m$}
\begin{algorithmic} 
\REQUIRE $n \geq 0 \vee x \neq 0 \vee z \geq 0 \vee m \neq 0$
\ENSURE $y = x^n + z^m$
\STATE $y \leftarrow 1$
\IF{$n < 0$}
  \STATE $N \leftarrow -n$
\ELSE
  \STATE $N \leftarrow n$
\ENDIF
\end{algorithmic}
\end{algorithm}
    \hypertarget{conclusions}{%
\section{Conclusions}\label{conclusions}}

Lorem ipsum dolor sit amet, consectetur adipiscing elit, sed do eiusmod
tempor incididunt ut labore et dolore magna aliqua. Ut enim ad minim
veniam, quis nostrud exercitation ullamco laboris nisi ut aliquip ex ea
commodo consequat. Duis aute irure dolor in reprehenderit in voluptate
velit esse cillum dolore eu fugiat nulla pariatur. Excepteur sint
occaecat cupidatat non proident, sunt in culpa qui officia deserunt
mollit anim id est laborum.


    % Add a bibliography block to the postdoc
    
    
\bibliographystyle{unsrt}
\bibliography{references}

    
    \end{document}
